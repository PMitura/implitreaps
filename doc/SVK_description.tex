\documentclass[a4paper, 12pt]{article}
\usepackage[sc]{mathpazo}
\usepackage[T1]{fontenc}	% Diacritic (output - selectable, hyphenation) 
\usepackage[utf8]{inputenc}	% Diacritic (input - the text) 
\usepackage[slovak]{babel}	% Typographical rules
\usepackage{array}			% Layouting
\usepackage{url}			% URL
\usepackage{graphicx}		% Figures
\usepackage{float}			% Figure positioning
\usepackage{booktabs}		% Table appearance

\usepackage[figurename=Obrázok]{caption}        % Margin styling
\usepackage[top=70pt, bottom=70pt, left=70pt, right=70pt]{geometry}
\usepackage[usenames, dvipsnames]{color} 
\usepackage{adjustbox}
\usepackage[binary-units=true]{siunitx}
\usepackage{enumitem}
\usepackage{pgfplots}
\usepackage[hidelinks]{hyperref}

\usepackage{amsthm}
\usepackage{amsmath}
\usepackage{amssymb}
\usepackage{enumerate}
\usepackage{thmtools}

\theoremstyle{definition}
\newtheorem*{otazka}{Otázka}
\newtheorem{definicia}{Definícia}
\newtheorem{lemma}{Lemma}
\newtheorem{dosl}{Dôsledok}
\newtheorem{priklad}{Príklad}
\newtheorem{veta}{Veta}

\declaretheoremstyle[
    spaceabove=6pt,
    spacebelow=6pt,
    headfont=\normalfont\itshape,
    qed=\qedsymbol,
    headpunct=.
]{proofstyle}
\declaretheorem[style=proofstyle,unnumbered, name=Dôkaz]{dokaz}

\pgfplotsset{compat=1.13}

% remove chapter headers
\usepackage{titlesec}
\titleformat{\chapter}{\normalfont\bfseries}{}{0pt}{\Huge}

% Morass' listing setup
\definecolor{myBlue}{RGB}{0,0,255}
\definecolor{alBlue}{RGB}{0,170,0}
\definecolor{myBrown}{RGB}{180,120,0}
\definecolor{myGreen}{RGB}{0,99,0}
\definecolor{myAzure}{RGB}{180,20,180}
\usepackage[formats]{listings}
\usepackage{listings}
\lstset{
    literate={~} {$\sim$}{1},
    language=C++,
    keywordstyle=\color{myBlue},
    commentstyle=\color{Gray},
    stringstyle=\color{myBrown},
    numberstyle=\ttfamily\color{Gray},
    emph={include,define},
    emphstyle={\color{myGreen}},
    keywordstyle=[2]\color{myAzure},
    keywords=[2]{string},
    basicstyle=\footnotesize\ttfamily,
    frame=bt,
    numbers=left,
    stepnumber=1,
    breaklines=true,
    emph=[2]{string,ll,ii,vi,vii,set,priority_queue,M_PI},
    emphstyle=[2]{\color{myAzure}},
    emph=[3]{F,PB,FF,FOR,aa,bb,CL,D},
    emphstyle=[3]{\bfseries\color{alBlue}},
    breakatwhitespace=true,
    belowskip=1.5em,
    aboveskip=1.5em,
    columns=fullflexible,
    stepnumber=1,
    keepspaces=true
}

% define "struts", as suggested by Claudio Beccari in
%    a piece in TeX and TUG News, Vol. 2, 1993.
\newcommand\Tstrut{\rule{0pt}{2.6ex}}         % = `top' strut
\newcommand\Bstrut{\rule[-0.9ex]{0pt}{0pt}}   % = `bottom' strut

\begin{document}

\title{\textsc{\large{Semestrálny projekt MI-PAL, LS 2016/17}}\\
\vspace{1cm}
\large{Implementácia dátovej štruktúry}\\
\Huge{Implicit Treap}\\
\vspace{1cm}
\normalsize{Peter Mitura (\texttt{miturpet@fit.cvut.cz})}}
\date{}

\maketitle

\section{Treap}

Treap (názov zložený zo slov \emph{tree} a \emph{heap}) je stromová dátová štruktúra,
ktorá je kombináciou binárneho vyhľadávacieho stromu (ďalej \emph{BVS}) a haldy.
Základy tejto štruktúry vychádzajú z \emph{kartézskych stromov}
\cite{cartesian} -- binárnych stromov postavených nad poľom, ktoré spĺňajú
haldovú podmienku a sú usporiadané tak, že ich in-order priechod vypíše
originálne pole.

Treapy sú rozšírením tejto myšlienky. Tiež sú to binárne stromy, v každom
svojom vrchole ale ukladajú dve hodnoty: \emph{kľúč} a \emph{prioritu}, pri čom
simultánne sa udržujú zoradené ako BVS podľa kľúča a ako minimová halda podľa
priority.  Nakoľko sa najčastejšie používajú ako implementácia abstraktného
dátového typu \emph{množina} (tzn. efektívna implementácia funkcií
\emph{Insert}, \emph{Find} a \emph{Delete}), kľúče vrcholov je logické využiť
na uloženie prvkov v množine. Priority potom slúžia jedine k vyvažovaniu
stromu, pri čom ďalej ukážeme, že ak ich budeme voliť náhodne z rovnomerného
rozloženia nad dostatočne veľkým rozsahom, strom získa z hľadiska efektivity
operácií nad kľúčmi zaujímavé vlastnosti.

Treap v tejto podobe bol po prvý krát popísaný Aragonom a Seidlom v roku 1989
\cite{treaps}, s určitými rozdielmi v názvosloví.
\footnote{Aragon a Seidel
pri pojme \emph{Treap} nešpecifikovali spôsob voľby priority, náhodnú hodnotu
jej dávali až v štruktúre ktorú pomenovali \emph{Randomized Search Tree} --
tento pojem sa ale neskôr začal využívať pre zložitejšiu dátovú štruktúru a
označenie Treap sa zaužíval pre verziu s náhodnou voľbou priority.}
Základné operácie, podporované touto štruktúrou zodpovedajú, ako už bolo
zmienené, abstraktnému dátovému typu množina:

\begin{itemize}
    \item $Insert(x)$ -- vloží prvok $x$ do stromu
    \item $Find(x)$ -- zistí, či sa prvok $x$ nachádza v strome
    \item $Delete(x)$ -- zmaže prvok $x$ zo stromu
\end{itemize}

Podobne ako u iných BVS, aj tu od univerza vkladaných prvkov požadujeme, aby
tvorilo úplne usporiadanú množinu a reláciu usporiadania je nutné definovať pri
inicializácií štruktúry (štandardne v implementácií využívame funkciu
\texttt{std::less}, ak je dostupná). Základnou myšlienkou potom je, že všetky
operácie implementujeme analogicky ako v bežnom BVS, s jediným rozdielom
že v nich chceme zachovať haldové usporiadanie podľa priority bez zvýšenia
asymptotickej časovej zložitosti.

Ak by sa nám to podarilo (konkrétnu implementáciu popíšeme nižšie), zložitosť
operácií nad stromom $T$ by dosiahla $\mathcal{O}(h(T))$, kde $h(T)$ je hĺbka
daného stromu. Tá by v nevyvažovanom BVS mohla dosiahnúť až
$\mathcal{O}(n)$,
kde $n$ je počet prvkov v strome, čo pre nás rozhodne nie je priaznivé. Kľúčové
pre celú štruktúru je teda nasledujúce tvrdenie:

\begin{veta}
    Ak je priorita každého vrchola nezávislá, rovnomerne rozložená spojitá
    náhodná veličina, stredná hodnota hĺbky \emph{ľubovoľného} vrchola v strome
    je $\mathcal{O}(\log n)$, kde $n$ je počet vrcholov v strome.
\end{veta}

Pri dôkaze tejto vety sa inšpirujeme postupom z kurzu na Univerzite v Illinois
\cite{erickson},
ktorý je do istej miery prehľadnejší než dôkaz v originálnom článku. 

Počet vrcholov v strome budeme naďalej označovať ako $n$. Ďalej si označme
vrchol s $k$-tým najmenším kľúčom ako $x_k$ a počet vrcholov. Zároveň si
zavedieme sadu premenných $A^i_k$, ktorá bude indikovať či je vrchol $x_i$
predkom vrchola $x_k$ (predka zavádzame tak, že vrchol nie je predkom sám
sebe):

$$
A_k^i = 
\begin{cases}
    \text{$1$\qquad ak je $x_i$ predkom $x_k$} \\
    \text{$0$\qquad inak}
\end{cases}
$$

Pomocou indikátora $A$ dokážeme jednoducho vyjadriť hĺbku vrchola $h(x_k)$:

\begin{equation}
h(x_k) = \sum_{i=1}^n A^i_k
\end{equation}

Takže strednú hodnotu hĺbky vrchola môžeme označiť ako

\begin{equation}
\label{etop}
\mathbb{E}(d(x_k)) = \sum_{i=1}^n P(A^i_k = 1)
\end{equation}

Veta 1 o tejto hodnote tvrdí, že je rovná $\mathcal{O}(\log n)$. Pre overenie
tejto rovnosti si zavedieme ešte jedno značenie, $X(i, k)$ bude označovať
množinu vrcholov $\{x_i, x_{i+1}, \ldots, x_k\}$ alebo $\{x_k, x_{k-1}, \ldots,
x_i\}$ podľa toho či $i < k$ alebo $i > k$.

\begin{lemma}
    Pre každé $i \neq k$ platí, že $x_i$ je predkom $x_k$ vtedy a len vtedy,
    keď má $x_i$ najnižšiu prioritu spomedzi vrcholov v $X(i, k)$.
\end{lemma}

\begin{dokaz}
    V prípadoch kedy je $x_i$ alebo $x_k$ koreňom stromu je rovno vidieť, že
    lemma platí ($x_i$ resp. $x_k$ má vtedy najnižšiu prioritu z celého
    stromu), obmedzíme sa teda na prípad kedy je koreňom $x_j$ pre nejaké $i
    \neq j \neq k$. Potom môžeme rozlíšiť dve možnosti: $x_i$ a $x_k$ sú buď
    v rovnakom alebo rozdielnom podstrome $x_j$.

    Ak sú v rôznom podstrome, tak $x_j$ určite patrí do $X(i, k)$ (jedná sa o
    BVS, takže jedna hodnota z dvojice $i$, $k$ musí byť menšia než $j$, a
    jedna väčšia než $j$). Priorita $x_j$ je ale najnižšia z celého stromu,
    takže aj z $X(i, k)$, čo by podľa lemmy malo znamenať že $x_i$ nie je
    predkom $x_k$ (a nakoľko sú v rôznych podstromoch, je to aj pravda).

    Pre $x_i$ a $x_k$ v jednom podstrome môžeme uvažovať len daný podstrom a
    v ňom postupovať analogicky ako v prípadoch vyššie (rekurzívne).
\end{dokaz}

Pravdepodobnosť že vrchol $x_i$ je predkom $x_k$ je teda rovnaká, ako
pravdepodobnosť že $x_i$ má minimálnu prioritu z $X(i, k)$ a nakoľko je táto
šanca pre všetky vrcholy z daného rozsahu rovnaká, môžeme ju jednoducho
vyjadriť ako:

$$
P(A^i_k = 1) = 
\left\{\begin{array}{lr}
    \vspace{1ex}
    \dfrac{1}{k-i+1} & \text{ak $i < k$} \\
    \vspace{1ex}
    \dfrac{1}{i-k+1} & \text{ak $i > k$} \\
    0 & \text{ak $i = k$}
\end{array}
$$

Zostáva už len vyjadriť hodnotu strednej hodnoty $\mathbb{E}(h(x_k))$. Z
\ref{etop} máme jej vzťah k $P(A^i_k)$, ich sumu môžeme rozdeliť na 2 časti
podľa vzťahu medzi $i$ a $k$:

\begin{equation}
    \sum_{i=1}^n P(A^i_k = 1) = \sum_{i=1}^{k-1} \frac{1}{k-i+1} +
    \sum_{i=k+1}^{n} \frac{1}{i-k+1}
\end{equation}

\begin{center}
\emph{(prvá suma je prípad s $i < k$, druhá $i > k$)}
\end{center}

Tieto sumy ďalej môžeme previesť na harmonické rady, pre ktoré máme známe
odhady:

\begin{equation}
    \sum_{i=1}^{k-1} \frac{1}{k-i+1} + \sum_{i=k+1}^{n} \frac{1}{i-k+1}
    = \sum_{j=2}^{k} \frac{1}{j} + \sum_{j=2}^{n-k} \frac{1}{j}
    = H_k - 1 + H_{n-k} - 1
\end{equation}
\begin{equation}
    H_k - 1 + H_{n-k} - 1 < \ln k + \ln(n-k) - 2 < 2 \ln n - 2
\end{equation}
čo asymptoticky dáva
\begin{equation}
    \mathbb{E}(d(x_k)) < 2 \ln n - 2 = \mathcal{O}(\log n)
\end{equation}
čím je veta 1 dokázaná.

\subsection{Implementácia operácií}


\begin{thebibliography}{9}
\bibitem{cartesian}
    J. Vuillemin, \uv{A unifying look at data structures}, Communications of the ACM,
        vol. 23, no. 4. Association for Computing Machinery (ACM), pp. 229–239,
        1. apríla 1980.
\bibitem{treaps} 
    C. R. Aragon a R. G. Seidel, \uv{Randomized search trees}, 30th Annual
        Symposium on Foundations of Computer Science. IEEE, 1989.
\bibitem{erickson}
    J. Erickson, \uv{Randomized Binary Search Trees}, 2013. [Online]. Dostupné
        z:
        \url{http://jeffe.cs.illinois.edu/teaching/algorithms/notes/10-treaps.pdf}.
        [prístup 9. mája 2017].
\end{thebibliography}
\end{document}
